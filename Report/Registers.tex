%  Registers.tex
%  Document created by seblovett on seblovett-Ubuntu
%  Date created: Tue 01 Apr 2014 20:10:32 BST
%  <+Last Edited: Thu 03 Apr 2014 12:21:48 BST by hl13g10 on octopus +>

\section{Registers}\label{sect:regs}

\todo[inline]{Design}
The allocation in the instruction set allows for up to 16 general purpose registers. 
In the program, discussed in section \ref{sect:prog}, only 11 registers are used. 
Six are used to store the constants, two for the initial vector, two for the result and a temporary register.
To save RAM, only the required registers are implemented.
This still requires a four bit address and attempting to address the registers above the valid range would result in undefined behaviour.

The registers were implemented using the Synchronous RAM.
This, at the expense of performance, utilised the on chip SRAM blocks. 
The design was parametrised to allow the data and address width and number of registers to be easily changed. 

\todo[inline]{Explain Test Bench}
\todo[inline]{Simulation Results}
\todo[inline]{Synthesis}
