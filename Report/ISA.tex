% ISA.tex
%  Document created by seblovett on seblovett-Ubuntu
%  Date created: Tue 01 Apr 2014 20:08:47 BST
%  <+Last Edited: Fri 04 Apr 2014 10:07:02 BST by hl13g10 on octopus +>


\section{Instruction Set}

%Discuss what instructions are implemented
In total, ten instructions are implemented in the design.
There are fully documented in table \ref{tab:isa}. 


\begin{table}
\caption{Instruction Set and their operations}
\label{tab:isa}
\begin{tabular}{ccc} \hline 
Instruction & Acronym & Operation \\ \hline
Store Switches & STSW Rn & Rn $\leftarrow$ Switches[7:0] \\
Store Accumulator & STACC Rn & Rn $\leftarrow$ Acc \\
Wait while 0 & WAIT0 & if(Switches[8] == 1) Pc++ \\
Wait while 1 & WAIT1 & if(Switches[8] == 0) Pc++ \\
Absolute Jump & JMPA Imm & PC $\leftarrow$ Acc + Imm \\
Load Upper Immediate & LUI Imm & Acc $\leftarrow$ {Imm, 4'b0000} \\
Add Immediate & ADDI Imm & Acc $\leftarrow$ Acc + Imm \\
Add Register & ADD Rn & Acc $\leftarrow$ Acc + Rn \\
Multiply Register & MULT Rn & Acc $\leftarrow$ Acc * Rn \\
Pass Register & PASSA Rn & Acc $\leftarrow$ Rn \\ \hline
\end{tabular}
\end{table}


\subsection{Arithmetic Instructions}

Only three instructions implemented conduct arithmetic operations, \texttt{ADD}, \texttt{ADDI} and \texttt{MULT}. 
These are all signed arithmetic and are the only functions needed to compute the affine transform. 
\texttt{ADDI} is the addition of an unsigned immediate. 
No subtract immediate instruction was implemented to reduce the size of the ALU and no sign extension is needed for the immediate. 

\subsection{Accumulator and Register Instructions}
%LUI, STSW, STACC, PASSA

The switches provide the main data input to the processor. 
A store switches instruction is implemented to directly store to a register. 
Load upper immediate is used to load constants from the program memory to the accumulator. 
Used in conjunction with the \texttt{ADDI} instruction, any 8 bit value can be loaded to the accumulator in two instructions. 
This can then be used to make a negative number to use for an arithmetic function.

The register-accumulator register, although allows for only one operand, must also include extra instructions to move data.
The instructions needed are a pass to accumulator (\texttt{PASSA}), to write a value from a register to the accumulator, with no modification. 
A store accumulator instruction is also needed to write back the data to a register. 

\subsection{Control Instructions}

The affine transform does not involve any decisions based on the data. 
The only required control flow is during the handshaking protocol. 
As this is reliant on the value of Switches[8], this is accomplished by two instructions, \texttt{WAIT0} and \texttt{WAIT1}. 
These will only allow the program counter to increment if the value of Swithes[8] is correct.
This means no flags need to be used in the design, and no conditional branches. 

An absolute jump is also implemented. 
It is used once the affine transform is complete to return to the start of the loop.
This replaces the value of the program counter with what is stored in the accumulator plus a four bit immediate. 
It means that the processor can jump to any eight bit memory location in two instructions.


\subsection{Instruction Format}

A total of ten instructions have been implemented. 
Combined with the use of a four bit immediate, this lends the instruction set to use an eight bit instruction.
This is summarised in table \ref{tab:instruction}.
A simple instruction format lends itself to simple decoding. 
A maximum of 16 registers (assuming no dummy register) can be used.

\begin{table}
\caption{The instruction format used}
\label{tab:instruction}
\centering
\begin{tabular}{|p{1cm}|p{1cm}|p{1cm}|p{1cm}|p{1cm}|p{1cm}|p{1cm}|p{1cm}|p{1cm}|} \hline
Bit & 7 & 6 & 5 & 4 & 3 & 2 & 1 & 0 \\ \hline
Use & \multicolumn{4}{|c|}{Opcode} & \multicolumn{4}{|c|}{Immediate / Register} \\ \hline
\end{tabular}
\end{table}
