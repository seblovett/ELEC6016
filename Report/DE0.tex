%  DE0.tex
%  Document created by seblovett on seblovett-Ubuntu
%  Date created: Tue 01 Apr 2014 20:11:21 BST
%  <+Last Edited: Fri 04 Apr 2014 11:32:14 BST by hl13g10 on octopus +>

\section{DE0 Implementation}



%\todo[inline]{Use of the slow clock}
A counter was used to slow down the $50MHz$ clock supplied from the board. 
This slowed the processor down to a human usable speed.
The added logic from this is not included in the cost function.


%\todo[inline]{Demo define to allow for easier use during demo.}
To aid the demonstration, a small amount of extra logic is used to show if the opcode is a \texttt{WAIT} instruction. 
This can be turned on or off by used of a \textit{`define} statement declared in the \textit{options.sv} file.
It provides an insight in to the current state of the processor and if it is waiting for an input.

This is accomplished by the code seen in listing \ref{lstleds}.
The logic added by here is not included in the overall cost function and is added to aid the use. 


\lstinputlisting[style=sverilog, firstline=30, lastline=39,caption={Extra LED logic to show when the processor is executing a \texttt{WAIT} instruction.},frame=single,label=lstleds]{../Implementation/cpu.sv}

\todo[inline]{Issues encounterd}

Program Memory not allowing to be small. 



\todo[inline]{Synthesised logic. Put some sexy figures in here}



