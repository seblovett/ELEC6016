%  ALU.tex
%  Document created by seblovett on seblovett-Ubuntu
%  Date created: Tue 01 Apr 2014 20:10:56 BST
%  <+Last Edited: Thu 03 Apr 2014 12:43:11 BST by hl13g10 on octopus +>


\section{ALU}

The arithmetic logic unit (ALU) is a key part of a processor. 
It is a combinational block of memory, responsible for the arithmetic and logic operations on the data.
The ALU in this processor has three operations on two operands.
No flags, such as carry or zero, are implemented either as no conditional branches are needed.

The ALU supports the operations in table \ref{tab:aluops}.
The operation encodings are discussed in section \ref{sect:controller}.
Due to the use of fixed point notation, the integer result of the multiplication is located at bits [14:7] of the 16 bit result.
To correctly synthesised combinational logic, all inputs must have a defined output.
The ALU was smallest in size when the A function was repeated in the redundant state.

\begin{table}
\caption{ALU Operations supported}
\label{tab:aluops}
\begin{tabular}{ccc}
Operation & Explanation & Encoding (binary)\\ 
A & Result = Operand1 & 00, 10 \\
Add & Result = Operand1 + Operand2 & 01 \\
Multiply & Result = (Operand1 $\times$ Operand2)[14:7] & 11 \\ 
\end{tabular}
\end{table}


The Cyclone IV FPGA has a number of embedded multipliers. 
One of these is utilised within the ALU to conduct signed multiplication. 
This is done by defining the inputs as signed, and using a combinational assign statement.
The Quartus tool then recognises this and synthesises the design using the embedded multipliers.
The multiplier cell also includes 3 registers which are not utilised in this design.


%\todo[inline]{Operations implemented}
%\todo[inline]{Embedded multiplier(s)}
\todo[inline]{Explain Test bench}
\todo[inline]{Simulation results}
\todo[inline]{Synthesis}

