%  Conclusion.tex
%  Document created by seblovett on seblovett-Ubuntu
%  Date created: Tue 01 Apr 2014 20:11:41 BST
%  <+Last Edited: Fri 04 Apr 2014 13:58:12 BST by hl13g10 on octopus +>


\section{Conclusion}

%\todo[inline]{Which objectives in intro have been done. }
The final processor satisfies the specification and the objectives highlights in section \ref{sect:intro}.
The processor is a register-accumulator based, MIPS-esk architecture. 
It implements ten instructions in total to calculate the affine transform of 8 bit data.

%\todo[inline]{Cost figure}
The total cost of the processor is 72.
A full breakdown of the costs is seen in table \ref{tab:costs}.
A large amount of the cost is located in the datapath. 
This is due to the multiplexors needed to direct the data around. 
A more compact design could be achieved by aiming to minimize these elements. 
The two registers in the datapath also contribute a fair amount to the cost. 
These are a compromise and by using a register-register architecture, the decoding logic is increased and a larger instruction is needed.
Overall, the design is a success, passing thorough testing and achieves the initial goals. 


\begin{table}
\caption{Break down of the costs by module. Costs do not include any instances made by the module.}
\label{tab:costs}
\begin{tabular}{cccc} \hline
Module		& Logic Elements	& Memory (bits)	& Embedded Multipliers \\ \hline
Control		& 9			& 0		& 0	\\
Datapath	& 24			& 0		& 0	\\
Registers	& 0			& 88		& 0	\\
ALU		& 21			& 0		& 1	\\
ROM		& 0			& 512		& 0 	\\ \hline
Total 		& 54			& 600		& 1 	\\ \hline
\end{tabular}
\end{table}

%\todo[inline]{General conclusion }

